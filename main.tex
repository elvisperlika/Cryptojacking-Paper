\documentclass[12pt,a4paper]{article}
\usepackage{textcomp, gensymb}
\usepackage[italian]{babel}
\usepackage{newlfont}
\usepackage{gensymb}
\usepackage{hyperref}
\usepackage{graphics}

\textwidth=450pt\oddsidemargin=0pt
\begin{document}
\begin{titlepage}
\begin{center}
\rule[0.5cm]{15.8cm}{0.6mm}
{\small{\bf Relazione di Crittografia }}
\end{center}
\vspace{15mm}
\begin{center}
{\LARGE{\bf Cryptojacking:} \\ 
\vspace{3mm}
{\bf quando il tuo Computer lavora per qaualcun altro.}}
\end{center}
\vspace{35mm}
\par
\noindent
\begin{center}
{\large{\bf
Elvis Perlika}}
\end{center}
\begin{center}
{\large{\bf
0000970373}}
\end{center}
\hfill

\vspace{70mm}
\begin{center}
{\large{\bf Corso di Crittografia \\ 
A.A. 2023-2024 \\
Prof. Luciano Margara}}
\end{center}
\end{titlepage}

\newpage

\tableofcontents

\newpage

\section{Introduzione}
\subsection{Definizione}

Il Cryptojacking è una forma di attacco informatico che sfrutta la potenza 
di calcolo di un utente, senza che esso ne sia consapevole, per minare criptovalute. \cite{CSO}

Gli Hacker hanno come obbiettivo quello di prendere il controllo del maggior numero 
possibile di sistemi con l'obbiettivo di minare quante più criptovalute, illecitamente. 
Questo sistema di hacking non punta unicamente la classica utenza di Personal Computer ma 
cerca di sfruttare anche le risorse di Server e infrastrutture Cloud e in generale 
ogni tipologia di sistema computazione con un accesso alla rete Internet.

La caratteristica fondamentale di questo malware è far sì che la vittima sia 
ignara dei processi in background, che si occupano di minare, e permettergli di usare 
la propria macchina normalmente. Ovviamente a discapito di un sovraccarico della macchina 
e conseguente surriscaldamento, presenza di lag, maggior consumo elettrico 
(che nel caso di servizi Server o Cloud porta ad avere fatture particolarmente elevate) 
e riduzione delle performance generali.

\subsection{Storia}

Una delle prime forme di cryptojacking è stata scoperta nel Giugno 2011, quando
l'azienda Symantec Corporation iniziò a sospettare che le botnet \footnote{Una botnet è
un gruppo di dispositivi connessi a Internet , ognuno dei quali esegue uno o più
bot . Le botnet possono essere utilizzate per eseguire attacchi DDoS (
distributed denial-of-service ), rubare dati, [1] inviare spam e consentire
all'aggressore di accedere al dispositivo e alla sua connessione. Il
proprietario può controllare la botnet utilizzando un software di comando e
controllo (C\&C). [2] La parola "botnet" è una parola risultata dalla unione delle parole "
robot " e " network ". Il termine è solitamente utilizzato con una connotazione
negativa o malevola.\cite{botnet}} potessero minare Bitcoin segretamente, sebbene la GPU 
di una sola macchina impiegherebbe molto tempo per minare una transizione in criptovalute, 
utilizzando una grande quantità di macchine si riesce a suddividere il lavoro e ridurre il tempo.

Una serie di attacchi rilevanti di cryptojacking sono stati scoperti dal 2011 al
2021. L'ultimo è relativo al "2021 Microsoft Exchange Server data breach"
\cite{zero-day}, tale breccia, creata nel Gennaio 2021 ha permesso numerosi
attacchi tra qui diversi di tipo cryptojacking.

Il cryptojacking è emerso come una minaccia significativa nel campo della
cybersecurity intorno al 2017, con l'introduzione di Coinhive, dismesso poi a
Marzo 2019 era un servizio di mining di criptovalute attraverso i browser web,
che andava a utilizzare parte o tutta la potenza di calcolo per minare
criptovalute Monero\footnote{Un particolare tipo di criptovaluta che si presta
particolarmente bene al mining utilizzando la CPU e caratterizzato da una
blockchain con tecnologie di miglioramento della privacy per offuscare le
transazioni per ottenere l'anonimato e la fungibilità.\cite{Monero}} mentre l'utente navigava
il sito. 

\section{Come funziona}
Il mining, cioè il processo che Bitcoin e altre cripto valute utilizzano per
coniare virtualmente nuove monete digitali e certificare le transazioni, usando
le relative monete, è completamente lecito. 

Nel dettaglio troviamo vaste reti decentralizzate di computer in tutto il mondo
che verificano e proteggono le blockchain, ovvero i registri virtuali che
documentano le transazioni di criptovalute. In cambio del contributo della loro
potenza di elaborazione, l'utente del computer della rete che per primo risolve
i calcoli complessi dovuti alla certificazione della transizione viene premiato
con nuove monete. Si tratta di un circolo virtuoso: i minatori mantengono e
tutelano la blockchain, la blockchain assegna le monete, le monete fungono da
incentivo ai minatori per continuare a mantenere la blockchain. Il mining è
l'unico modo per rilasciare nuove cripto monete nella rete ed è un processo che
richiede molta potenza di calcolo con un effort inversamente proporzionale al
mining effettuato portando così ad un aumento della difficoltà di mining e ad
una conseguente crescita dei costi.

Il cryptojacking sfrutta questo processo, ma in modo illecito. Gli hacker
inseriscono codice malevolo nei siti web o nei messaggi di posta elettronica che
infettano i computer delle vittime e li trasformano in macchine per il mining 
riducendo i costi e aumentando i guadagni.

\section{Metodi di attacco}
I metodi per attaccare un sistema con il cryptojacking sono molteplici e
variano a seconda del tipo di sistema che si vuole attaccare. I metodi più
comuni sono:

\subsection{Attaccare diretamente i Personal Computer}
Attaccare uno o più PC è il classico metodo per creare un sistema di
cryptojacking. Tipicamente l'hacker riesce ad iniettare il suo software 
di mining all'interno della macchina usando tecniche come:
\begin{itemize}
    \item Fileless malware
    \item Schemi di phishing
    \item Embedded di script malevoli al interno di siti o web app
\end{itemize}

Il modo più semplice con cui gli aggressori di cryptojacking possono rubare
risorse è inviare agli utenti un'e-mail dall'aspetto legittimo che li incoraggi
a fare clic su un collegamento che esegue il codice per inserire uno script di
cryptomining sul proprio computer. Funziona in background e invia i risultati
tramite un'infrastruttura di comando e controllo (C2\footnote{Command and
Control Infrastructure: anche conosciuto come C\&C o C2 è il set di strumenti e
tecniche che un un hacker utilizza per mantenere la commuicazione con il
computer precedentemente compresso.}).

In alternativa gli hacker possono sfruttare script all'interno dei siti, che eseguiti 
automaticamente dai browser, minano le cripto valute. Questo metodo è molto più
diffuso e meno invasivo rispetto al precedente, poiché non sscarica alcun codice nel dispositivo.

\subsection{Cercare server e dispositivi di rete vulnerabili}
I server sono un obbiettivo molto ambito per gli hacker, in quanto sono
dispositivi molto potenti e spesso connessi a Internet 24/7. Gli hacker possono
sfruttare vulnerabilità come Log4J\footnote{La vulnerabilità Log4j, conosciuta
anche come Log4Shell, è una vulnerabilità critica scoperta nella libreria di
registrazione Apache Log4j nel novembre del 2021. Sostanzialmente, Log4Shell
concede agli hacker il controllo totale dei dispositivi eseguendo versioni di
Log4j senza patch.\cite{Log4J}} per iniettare i propri sistemi di cryptojacking in queste
potenti macchine. Spesso i server compromessi vengono anche utilizzi come
potente per accedere con maggior semplicità ad altri
dispositivi per eseguire attacchi più complessi ed orizzontali.

\subsection{Attaccare il sistema di produzione di software}
Un altro metodo molto comune è quello di attaccare il sistema di seminare
repository open-souce nelle quali è stato iniettato il loro codice malevolo.
Grazie ai programmatori che utilizzano questi codici è possibile per 
gli hacker raggiungere un numero elevato di macchine e scalare velocemente 
il loro sistema di mining.
Una volta entrati nella macchina del programmatore, possono cercare di accedere
anche ai server, ai dispositivi di rete oppure ai servizi cloud ai quali esso è
connesso. In alternativa possono puntare a sub-iniettare questi script all'interno 
dei progetti che i programmatori stanno sviluppando.

\subsection{Fare leva sulle infrastrutture cloud}
Come per i server, anche le infrastrutture cloud sono un obbiettivo molto ambito
poiché permettono di effettuare computazioni ancora più veloci. Uno dei metodi
più comuni per farlo è scansionare le API dei container esposti e utilizzare
tale accesso per avviare il caricamento del software di mining sulle istanze dei
container o sui server cloud interessati. L'attacco è in genere automatizzato
con un software di scansione che cerca server accessibili alla rete Internet
pubblica con API esposte o che permettono l'accesso senza autenticazione. Come
per i server, gli aggressori sfruttano il cloud service violato ed attraverso lo
stesso puntano a raggiungere altre infrastrutture simili.
Questi sono gli attacchi più redditizi.
\\
L'aspetto rilevante, in tutti gli approcci sopra citati, è che gli hacker possano 
accedere a quante più macchine computazionali.

\section{Aspetti tecnici}

\section{Popolarità}
La popolarità è dovuta al potenziale guadagno, guadagno molto facile da crearsi
poiché per definizoione il cryptojacking punta a sfruttare risorse in possesso
di altri in modo gratuito. Così, anche considerando la volatilità delle cripto
valute, esempio principe BitCoin, i margini di guadagno sono abbastanza alti da rendere 
il crimine un vero e proprio business. 

\section{Prevenire}

\section{Bibliografia}
\begin{thebibliography}{9}

\bibitem{CSO}
\href{https://arc.net/l/quote/karbftmg}{Cryptojacking explained, CSO}
% \texttt{31-07-2024 15:00}

\bibitem{botnet}
\href{https://arc.net/l/quote/ftyxgxms}{Botnet, Wikipedia}
% \texttt{31-07-2024 13:00}

\bibitem{zero-day}
\href{https://arc.net/l/quote/golshtco}{2021 Microsoft Exchange Server data breach, Wikipedia}
% \texttt{01-08-2024 16:00}

\bibitem{Monero}
\href{https://arc.net/l/quote/jffmkeln}{Monero, Wikipedia}
% \texttt{01-08-2024 16:30}

\bibitem{Log4J}
\href{https://arc.net/l/quote/zjujxamu}{Log4J, IBM}
% \texttt{01-08-2024 18:30}


\end{thebibliography}


\end{document}